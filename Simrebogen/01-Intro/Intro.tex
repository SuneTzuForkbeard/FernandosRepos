\section*{Why do we need another fusion dance style?}
... Not to mention another martial art? \textit{Surely, any civilized major city in the Western World has a very diversified offering of both dance- and martial arts-classes, so can't we just sign up for one of those?}\\
Sure you can, and I strongly urge you to sign up for one or several of both and find a couple that you like. But that shouldn't stop you from \sovsing. Hopefully, those classes will only increase your \textit{rizz} and your skillset as a \sovseneer

\paragraph{}
\sovs is not another elaborated curriculum of neither dancing nor fighting, and there is no need to set up a Duolingo rutine to learn Spanish, Chinese or Japanese in order to \sovs\\
The fundamental principles of \sovs can - and more than likely \textit{will} - be applied to any style of dancing, but \sovs specifically refers to the fusion of \textbf{\textit{Cuban Salsa}} with the martial arts.\\
Whereas the type of \textit{dancing} to use as framework for the expressions of \sovs is fixed to Cuban Salsa, the choice of \textit{martial art} applied by the \sovseneer is completely free.

\section*{\sovs optimizes the mating ritual}
For as long as anyone can remember, the dancefloor has been the goto place for humans to perform initial steps of the mating ritual\footnote{During the last 10+ years, online dating has seen an incredible rise in popularity, but seeing as the author of this book has no experience what-so-ever in the world of online dating, we will base the discussion on data pre-dating the age of online dating. I'm sure it's just a fad anyways}, with very little change to the overall principles of interaction. Consider the following scene.

\begin{scene*}[In the club]
\dude is in the club looking for a compatible mate, when he locks eyes with \scactorf{Gal}, who is in the club with exactly the same intentions. Now, \scactorf{Gal} wants to make sure that the communication with a potential mate will work before indulging in a night of passion. She takes to the dance floor hoping to get a sense of \dude's physical expressiveness. \dude would like to engage in a night of passion with \scactorf{Gal}, but doesn't know the first thing about dancing. Being afraid that he will make a fool of himself on the dance-floor, he boosts the ol' self-confidence with a bottle of $37.5\%$ courage-potion from the bar and a couple of lines of concentrated coca leaves extract in the bathroom, before going to the dance floor and making a complete fool of himself.\\
Hellbent on proving his peak physical state to \scactorf{Gal}, he picks a fight with \dudetwo. \scactorf{Gal} get's upset by the foolish display, \dudetwo get's a couple of black eyes and a trip to the dentist. \dude doesn't get any.
\end{scene*}

\newpage

{\color{cfemale}{\subsection*{Benefits of \sovs for \textit{her}}}}

\subsubsection*{\sovs provides a clear and easy way to show disinterest}
Not interested in the person you are dancing with? Easy: \attknytter to the face. Is he having a hard time getting the message? \atttrykker to the chest, and off you go. Simple as that.

\subsubsection*{\sovs is built to test physical fitness}
Maybe interested in the person that you are dancing with, but want to ensure that he has the proper lower body endurance to keep up with you? Easy: go for a couple of \attlosser$\!$s to the thighs. If he is still standing before your shins tire from the testing, he'll probably do fine later on.

\subsubsection*{\sovs is perfect for showing interest}
Think your dancing partner is hot shit, and want to unambiguosly tell him that it's about time for the two of you to leave this place and head to more appropriate lodgings? Go for the take-down. That's a language that even the drunkest drunkard will understand. 


\newpage
{\color{cmale}{\subsection*{Benefits of \sovs for \textit{him}}}}

\subsubsection*{It is a lot more fun being punched by women than other dudes}
\begin{scene*}[In the dojo/mouh goon/gym]
\scactorm{Instructor}: \textit{Watch closely as I demonstrate this cool technique. [does cool technique]\\
Now you do! And focus on technique instead of beating each other into a pulp!}\\
\dude and \dudetwo: \textit{YES SENSEI/SIFU/SIR!}\\
\paragraph{}
$\cdots$ \textit{2 minutes later} $\cdots$\\
\paragraph{}
\dude and \dudetwo: \textit{[beating each other into a pulp]}\\
\scactorm{Instructor}: you are not practicing the technique!
\dude: \textit{Well, we \textbf{were} practicing the technique, but then \dudetwo punched me in the gut, and you know how there is something about \dudetwo that makes you just want to punch him in the face and well...}\\

\scactorm{Instructor}:
\begin{center}
\includegraphics[scale=0.5]{01-Intro/concerned-sifu}
\end{center}
\end{scene*}

I'm sure that anyone who has put more than 5 hours into the martial arts will reckognize the scenario above, and let's face it: it's a lot more fun to practice techniques with women than men. Unfortunately, in most places where martial arts is practiced, the woman:man ratio doesn't really favor this fact.\\
\sovs greatly remedies this by taking technique-practice onto the dance floor! 

\newpage
\subsubsection*{\sovs increases the vocabulary of movements on the dance floor}
Have you ever taken dance lessons? If so, you might have come across an instructor who has uttered such words as: \textit{...and then the man shows off his \textbf{machismo} by doing this movement [proceeds to perform movement that conveys very \textbf{little} machismo]}\\
How the hell are you supposed to express your \textit{inner caveman}, if you are only allowed to use a \textit{girlish} vocabulary? I bet you don't feel girlish when you are \textit{ducking/slipping} punches or taking \textit{shin-kicks} to the thighs during sparring, right?\\ 
Well, there you go!

\subsubsection*{\sovs let's you see another side of her before comitting to any type of \textit{tomfoolery}}
You can tell a lot about a woman by how she punches you in the face, but unless you have a really annoying personality you are not likely to see that side of her until you are knee-deep in dirty diapers, and by then it's sorta too late. To this end, \sovs is the perfect \textit{try before you buy}!

