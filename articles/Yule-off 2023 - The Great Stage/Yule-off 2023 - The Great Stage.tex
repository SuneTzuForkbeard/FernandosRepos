\documentclass[danish]{article}
\usepackage[T1]{fontenc}
%\usepackage[utf8]{inputenc}

\usepackage[danish]{babel}
%\usepackage[danish]{isodate}
\usepackage[dvipsnames,table]{xcolor}

\usepackage{ulem}
%https://mirrors.dotsrc.org/ctan/macros/luatex/latex/emoji/emoji-doc.pdf
%\usepackage{emoji}
\usepackage{amsmath,amsfonts,amssymb,amsthm}
\usepackage{multirow}
\usepackage{lilyglyphs} %https://www.ctan.org/pkg/lilyglyphs
\usepackage{boldline}
\usepackage{etoolbox}
\usepackage{xstring}
\usepackage{csquotes}
%\usepackage{datetime}
\usepackage{microtype}
%\usepackage{newspaper}
\usepackage{yfonts}  % used for the paper title font
\usepackage{fontspec}
%\usepackage{
%\date{\today}

%\SetPaperName{Solsikke Tidende}
%\SetPaperLocation{Santiago de Cali}
%\SetPaperSlogan{\clefG \hspace{0.15cm} \textbf{$Cipher_{Caesar}(_3)$} \hspace{0.15cm} \clefG}
%\SetPaperPrice{}

\DeclareFontFamily{LYG}{bigygoth}{}
\DeclareFontShape{LYG}{bigygoth}{m}{n}{<->s*[2.5]ygoth}{}

\setlength\topmargin{-48pt} 		% article default = -58pt
\setlength\headheight{0pt}  		% article default = 12pt
\setlength\headsep{34pt}		% article default = 25pt
\setlength\marginparwidth{-20pt}	% article default = 121pt
\setlength\textwidth{7.0in}		% article default = 418pt
\setlength\textheight{9.5in}		% article default = 296pt
\setlength\oddsidemargin{-30pt}



\def\papername{Solsikke Tidende}
\def\headername{Solsikke Tidende}   % because of the yfonts you may need both papername and headername
\def\paperlocation{København}
\def\musagsduhint{{\footnotesize \clefG} \hspace{0.15cm} \textbf{$Cipher_{Caesar}(4)$} \hspace{0.15cm} {\footnotesize \clefG}}
\newcommand\muzait{{\scriptsize\twoBeamedQuavers}\,}

\newcommand\muzaselection[3]{
  \paragraph{}
  {\fontfamily{lmr}\selectfont
  \textit{
    \muzait 
    {\Large\textbf{\StrLeft{#1}{1}}}\StrGobbleLeft{#1}{1}\, 
     \muzait} (#2) \\
  {\scriptsize
   \textit{
      \blockquote{#3}
    }
  }
}}

\def\paperprice{0 DKK}

\newcounter{volumeno}
\setcounter{volumeno}{67}
\newcounter{issueno}
\setcounter{issueno}{28}



\usepackage{times}
\usepackage{graphicx}
\usepackage{multicol}
\usepackage{picinpar}

\usepackage{lipsum}


%\colorlet{cfernando}{LightSteeleBlue3}
\definecolor{cfernando}{RGB}{38,40,74}
\definecolor{canna}{RGB}{164, 39, 168}
\definecolor{csune}{RGB}{125,21,21}
\definecolor{cmaria}{RGB}{245, 66, 120}
\definecolor{cmartin}{RGB}{168, 50, 60}
\definecolor{cother}{RGB}{103, 103, 152}
\definecolor{cdo}{RGB}{29, 222, 80}
\definecolor{cdont}{RGB}{219, 2, 42}
\definecolor{cmaybe}{RGB}{235, 183, 52}



\newcommand\sayanna[1]{
  {\fontfamily{lmr}\selectfont \color{canna}{\textit{- #1}}}
}

\newcommand\saymaria[1]{
  {\large
     {\setmainfont{Gabriola} \color{cmaria}{\textit{- #1}}}
  }
}

\newcommand\saymartin[1]{
  {\huge
     {\setmainfont{Segoe Script} \color{cmartin}{\textit{- #1}}}
  }
}


\newcommand\sayfernando[1]{
  {\fontfamily{lmr}\selectfont \color{cfernando}{\textit{- #1}}}
}

\newcommand\saysune[1]{
  {\setmainfont{Ink Free} \color{csune}{\textit{- #1}}}
}

\newcommand\sayother[1]{
  {\fontfamily{lmr}\selectfont \color{cother}{\textit{- #1}}}
}

\newcommand\saydrusilla[1]{
  {\setmainfont{Segoe Print} {\textit{- #1}}}
}


\newcommand\tabdesc[1]{
  {
     {\setmainfont{Impact} {\textit{#1}}}
  }
}


\newcommand\tabdo[1]{
  {
     {\setmainfont{Impact} \color{cdo}{\textit{#1}}}
  }
}

\newcommand\tabdont[1]{
  {
     {\setmainfont{Impact} \color{cdont}{\textit{#1}}}
  }
}

\newcommand\tabmaybe[1]{
  {
     {\setmainfont{Impact} \color{cmaybe}{\textit{#1}}}
  }
}


\newcommand\translatedfrom[1]{
 \begin{center}
  {\fontfamily{lmdh}\selectfont
  {\textit{oversat fra #1}}
  }
 \end{center}

}


\renewcommand{\maketitle}{
  \thispagestyle{empty}
  \vspace*{-40pt}
  \begin{center}
  \hfill
  {\textgoth
   {\huge 
     \usefont{LYG}{bigygoth}{m}{n} \papername
   }
  }\hfill%	
  \raisebox{12pt}{
   \textbf{
    \footnotesize 
    \paperlocation
   }
  }\\
  \vspace*{0.1in}
  \rule[0pt]{\textwidth}{0.5pt}\\
  {\small 
    VOL.\MakeUppercase{\roman{volumeno}}
    \ldots No. \arabic{issueno}
   } \hfill 
   \MakeUppercase{\small  19. december 2023} 
   \hfill {\small }\\
  \rule[6pt]{\textwidth}{1.2pt}
  \end{center}
  \pagestyle{plain}
}

\def\ps@plain{%
  \renewcommand\@oddfoot{}%					% empty recto footer
  \let\@evenfoot\@oddfoot						% empty verso footer
  \renewcommand\@evenhead
  {\parbox{\textwidth}{\vspace*{4pt}
  {\small VOL.\MakeUppercase{\roman{volumeno}}\ldots No.\arabic{issueno}}\hfill\normalfont\textbf{\headername}\quad\MakeUppercase{\textit\today}\hfill\textrm{\thepage}\\
  \rule{\textwidth}{0.5pt}
  \vspace*{12pt}}}%
  \let\@oddhead\@evenhead
}
		

\newcommand\headline[1]{
  {\fontfamily{lmdh}\selectfont
    \begin{center} #1\\ %
    \rule[3pt]{0.4\hsize}{0.5pt}\\ \end{center} \par
  }
}

\newcommand\byline[2]{
  {\fontfamily{lmdh}\selectfont
  \begin{center} #1 \\%
  {\footnotesize\bf af \MakeUppercase{#2}} \\ %
  \rule[3pt]{0.4\hsize}{0.5pt}\\ \end{center} \par
  }
}


\newcommand\closearticle{{\begin{center}\rule[6pt]{\hsize}{1pt}\vspace*{-16pt}
			\rule{\hsize}{0.5pt}\end{center}}}



\begin{document}


\maketitle
\fontfamily{phv}\selectfont

\begin{multicols}{2}
\byline{Tingene spidser til}{Fernándo Sanchez}
\subsection*{All Good Things come to an End}
\saymaria{Kan vi tage bare en tur mere?}\\
\sayfernando{Vi har været i gang i 8 timer! Tror du ikke at den falder af?}\\
\saymaria{Visse vasse... Her, tag 2 mere af de blå, og en powernap... Så går jeg ud og laver en lille snack til os, og så tager vi 2. halvvej bagefter}\\
\sayfernando{2.?}\\
\saymaria{Well...} sagde hun, iførte sig sin korte natkjole og gik imod lejlighedens lille køkken imens hun talte på fingrene. Imens nærmest besvimede jeg på sofaen som jeg indtil for 2 minutter havde brugt som rampe.\\
Jeg nåede at sove i omtrent så lang tid som det tager at tilberede en omgang beef nachos, for da jeg vågnede af lyden fra en meget larmende motorcykel der stoppede nede på gaden, var Maria på vej ind i stuen med en bakkefuld beef nachos ingredienser.\\
\saymaria{Ej for helvede, nu nåede jeg ikke at lave guaca'en} stressede hun, og skyndte sig tilbage ud i køkkenet.\\
Jeg fik en fornemmelse af at noget vigtigt skulle til at ske, som blev forstærket da Maria kom løbende tilbage ind i stuen med en skålfuld ingredienser til guacamole. Imens Maria havde haft travlt i køkkenet og jeg havde sovet, havde de blå piller også været flittige og nu var i hvert fald en lille del af mig lysvågen. Idet jeg rejste resten af kroppen op i sofaen, kunne jeg høre et kraftigt brøl fra trappen.

\saymartin{HVEM FANDEN HAR DU NU KNALDET MED MARIA?}\\
\sayfernando{Hvem er det?}\\
\saymaria{Det er min kæreste} sagde Maria, som nu havde taget stilling i lænestolen i hjørnet af stuen, og sad som skulle hendes yndlings TV-serie til at begynde.\\
\saymaria{Han er psykopat!} tilføjede hun med et kæmpe smil.\\\\
\sayfernando{Erhvervspsykopat?} spurgte jeg håbefuldt.\\
\saymaria{(NEJ!)} rystede hun på hovedet, med øjne der skinnede af forventning, og skovlede en håndfuld ostede nachos indenbords.\\
\sayfernando{Oh}\\

Med det samme blev vores lille tosomhed i lejligheden afbrudt, idet en kæmpe økse brød igennem hoveddøren og nogle sekunder efter kom en blodindsølet balrog i julemandskostume bragende igennem døren, svingende vildt omkring sig med en lige så blodindsmurt økse.\\

\saymartin{DU ER KRAFTEDEME EN DØD MAND, GADEDRENG!} sagde det enorme menneske og tonsede imod mig med øksen truende hævet over hovedet. I mellemtiden nåede Maria ét-eller-andet slags klimaks i lænestolen, og lod sig falde tilbage med et veltilfreds smil.\\

Jeg forestiller mig at du, kære læser, tilhører det bedre borgerskab og er udstyret med en over-gennemsnits begavelse, så jeg vil ikke tale ned til dig og forsøge at give en suspenseful beretning af begivenhederne der fulgte; eftersom jeg skriver disse linjer har du jo nok regnet ud at jeg overlevede mødet med den enorme fyr. Som en public service, vil jeg i stedet (i figur \ref{fig:diffs}) påpege nogle subtile forskelle imellem reglerne (eller mangel på samme) der gælder i det velkendte rituelle gadeslagsmål og kampen for overlevelse imod vilde økse-svingende voldspsykopater. 


\subsection*{In for a penny, in for a pound}

Anna ringede kort tid efter at jeg var spænet ud af lejligheden.\\

\sayanna{Hvor i alverden er du henne Fernando? Er du klar over hvilken dag det er i dag?}\\
\sayfernando{Jeg har vist stadigvæk en del at lære om de danske højtider, men mit bud er prøv-at slå-den-rare-colombianer-ihjel-med-kløveøkse-dag?} svarede jeg forpustet.\\
\sayanna{...}\\
\sayfernando{Hvad har jeg vundet?}\\
\sayanna{KUN INGENTING! OG FÅ SÅ FINGEREN UD OG KOM HERIND!}\\
\sayfernando{Så gerne! Lige nu løber jeg ad Peter Bangs vej med retning imod indre by. Jeg tror at jeg kan være der om 20 minutter.}\\
\sayanna{Ja ja, det er vel bedre end ingenting!}\\
\sayfernando{Måske 15... man løber hurtigt når man er helt nøgen og folk glor efter éns erek...}\\
\sayanna{AD! Du kan godt holde op lige med det samme! Du skal ikke komme herind!}\\
\sayanna{Du skal viderebringe en kort update vedrørende Yule-off 2023. Kan du klare det? Har du noget at skrive med?}\\
Jeg følte mig ikke vildt i overskud af ilt, og jeg tænkte at Anna nok skulle regne svaret på spørgsmål \# 2 ud, og for så vidt gælder spørgsmål \#1 ville hendes bud være ligeså godt som mit. 

\sayanna{Ja ja, her kommer update...}\\

Så kære læser, uden yderligere omsvøb, kommer her Annas update:

\begin{itemize}
  \item Den første finalist i Yule-off Musical Contest 2023 er fundet, idet ``Fairytale of New York'' idag (tirsdag) sikkert slog ``A natale puoi'' 8-4
  \item I morgen (onsdag) skal den sidste finalist findes, når klassens bad-boy \textbf{Suspect} inviterer til slåskamp i frikvarteret imod kæledæggen \textbf{Rasmus Seebach}
  \item I morgen (onsdag) finder vi formodentligt vinderen af såvel årets Team Contest som det individuelle klassement der er udsprunget i dets kølvand, når vi afvikler årets Kongeetape: ubegrænset Meowy Christmas frem til kl. 23:59
\end{itemize}

Som jeg sidder på Solsikke Tidendes redaktion og skriver disse linjer, er der et par ting der går op for mig:

\begin{itemize}
  \item toget er kørt for at mit forhold til fru Saxing nogensinde bliver normalt igen (men det har nu alligevel været ret medtaget siden julefrokosten 2022)
  \item selv med dagens begivenheder frisk i erindring, priser jeg mig lykkelig for at jeg står på denne side af skillelinjen mellem ``udsending fra pressen'' og ``kombatant i Yule-off 2023 - Team Contest''
\end{itemize}

Tanken på den dedikation der skal til for at hive sejren hjem i Yule-off 2023 - Team Contest, har inspireret mig til at se min ``spirituelle dannelsesrejse'' til ende, så det er med rystende hænder at jeg nu ringer til det nummer som CMO'en gav mig, som et naturligt næste skridt på rejsen.\\

\saydrusilla{Det er Drusilla}\\
\sayfernando{Jeg fik dit nummer af CMO'en...}\\
\saydrusilla{Det ved jeg ikke hvad er for noget}\\
\sayfernando{Jeg er på en spirituel dannelsesrejse...}\\
\saydrusilla{Spiser du hvidløg?}\\
\sayfernando{Det gør ingen!}\\
\saydrusilla{Så går du an! Mød mig foran Emdrupparkens Idrætsanlæg præcis kl. 21 i aften!}\\
\sayfernando{Det lyder hyggeligt}\\
\saydrusilla{... (lægger på)}\\


\end{multicols}

\begin{figure}
	\centering
\begin{tabular}{ | l | p{4cm}|  p{2cm} | } 
  \hline
  & \tabdesc{Gadeslagsmål} & \tabdesc{Julemand} \\ 
  \hline
  \hline
  \tabdesc{Slag, spark, tramp} & \tabdo{Go for it} & \tabdo{Go for it} \\
  \hline
  \tabdesc{Skalder, albuer, knæ} & \tabdo{Go for it} & \tabdo{Go for it} \\
  \hline
  \tabdesc{``Kender du ham der står bag dig?''} & \tabdo{Go for it} & \tabdo{Go for it} \\
  \hline
  \tabdesc{Skridtspark, finger-vrid} & \tabmaybe{Kun hvis du er ved at tabe} & \tabdo{Go for it} \\
  \hline
  \tabdesc{Kast dig på jorden og lad som om du spasmer} & \tabmaybe{Ikke hvis du kan undgå det} & \tabdo{Go for it} \\
  \hline
  \tabdesc{Angrib med kemisk-oprejste legemer først} & \tabdont{Lad være} & \tabdo{Go for it} \\
  \hline
  \tabdesc{Råb at ``bylderne smitter ved berøring''} & \tabdont{Lad være} & \tabdo{Go for it} \\
  \hline
\end{tabular}
	
\caption{Do's and don'ts i \textbf{gadeslagsmål} vs. \textbf{kamp for overlevelse imod psykopat-julemænd}}
\label{fig:diffs}
\end{figure}



\end{document}