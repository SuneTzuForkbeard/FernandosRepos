\documentclass[danish]{article}
\usepackage[T1]{fontenc}
%\usepackage[utf8]{inputenc}

\usepackage[danish]{babel}
%\usepackage[danish]{isodate}
\usepackage[dvipsnames,table]{xcolor}

\usepackage{ulem}
%https://mirrors.dotsrc.org/ctan/macros/luatex/latex/emoji/emoji-doc.pdf
%\usepackage{emoji}
\usepackage{amsmath,amsfonts,amssymb,amsthm}
\usepackage{multirow}
\usepackage{lilyglyphs} %https://www.ctan.org/pkg/lilyglyphs
\usepackage{boldline}
\usepackage{etoolbox}
\usepackage{xstring}
\usepackage{csquotes}
%\usepackage{datetime}
\usepackage{microtype}
%\usepackage{newspaper}
\usepackage{yfonts}  % used for the paper title font
\usepackage{fontspec}
%\usepackage{
%\date{\today}

%\SetPaperName{Solsikke Tidende}
%\SetPaperLocation{Santiago de Cali}
%\SetPaperSlogan{\clefG \hspace{0.15cm} \textbf{$Cipher_{Caesar}(_3)$} \hspace{0.15cm} \clefG}
%\SetPaperPrice{}

\DeclareFontFamily{LYG}{bigygoth}{}
\DeclareFontShape{LYG}{bigygoth}{m}{n}{<->s*[2.5]ygoth}{}

\setlength\topmargin{-48pt} 		% article default = -58pt
\setlength\headheight{0pt}  		% article default = 12pt
\setlength\headsep{34pt}		% article default = 25pt
\setlength\marginparwidth{-20pt}	% article default = 121pt
\setlength\textwidth{7.0in}		% article default = 418pt
\setlength\textheight{9.5in}		% article default = 296pt
\setlength\oddsidemargin{-30pt}



\def\papername{Solsikke Tidende}
\def\headername{Solsikke Tidende}   % because of the yfonts you may need both papername and headername
\def\paperlocation{København}
\def\musagsduhint{{\footnotesize \clefG} \hspace{0.15cm} \textbf{$Cipher_{Caesar}(4)$} \hspace{0.15cm} {\footnotesize \clefG}}
\newcommand\muzait{{\scriptsize\twoBeamedQuavers}\,}

\newcommand\muzaselection[3]{
  \paragraph{}
  {\fontfamily{lmr}\selectfont
  \textit{
    \muzait 
    {\Large\textbf{\StrLeft{#1}{1}}}\StrGobbleLeft{#1}{1}\, 
     \muzait} (#2) \\
  {\scriptsize
   \textit{
      \blockquote{#3}
    }
  }
}}

\def\paperprice{0 DKK}

\newcounter{volumeno}
\setcounter{volumeno}{67}
\newcounter{issueno}
\setcounter{issueno}{28}



\usepackage{times}
\usepackage{graphicx}
\usepackage{multicol}
\usepackage{picinpar}

\usepackage{lipsum}


%\colorlet{cfernando}{LightSteeleBlue3}
\definecolor{cfernando}{RGB}{38,40,74}
\definecolor{canna}{RGB}{164, 39, 168}
\definecolor{csune}{RGB}{125,21,21}
\definecolor{cmaria}{RGB}{245, 66, 120}
\definecolor{cmartin}{RGB}{168, 50, 60}
\definecolor{cother}{RGB}{103, 103, 152}
\definecolor{cdo}{RGB}{29, 222, 80}
\definecolor{cdont}{RGB}{219, 2, 42}


\newcommand\sayanna[1]{
  {\fontfamily{lmr}\selectfont \color{canna}{\textit{- #1}}}
}

\newcommand\saymaria[1]{
  {\large
     {\setmainfont{Gabriola} \color{cmaria}{\textit{- #1}}}
  }
}

\newcommand\saymartin[1]{
  {\huge
     {\setmainfont{Segoe Script} \color{cmartin}{\textit{- #1}}}
  }
}


\newcommand\sayfernando[1]{
  {\fontfamily{lmr}\selectfont \color{cfernando}{\textit{- #1}}}
}

\newcommand\saysune[1]{
  {\setmainfont{Ink Free} \color{csune}{\textit{- #1}}}
}

\newcommand\sayother[1]{
  {\fontfamily{lmr}\selectfont \color{cother}{\textit{- #1}}}
}

\newcommand\tabdo[1]{
  {
     {\setmainfont{Impact} \color{cdo}{\textit{#1}}}
  }
}

\newcommand\tabdont[1]{
  {
     {\setmainfont{Impact} \color{cdont}{\textit{#1}}}
  }
}

\newcommand\translatedfrom[1]{
 \begin{center}
  {\fontfamily{lmdh}\selectfont
  {\textit{oversat fra #1}}
  }
 \end{center}

}


\renewcommand{\maketitle}{
  \thispagestyle{empty}
  \vspace*{-40pt}
  \begin{center}
  \hfill
  {\textgoth
   {\huge 
     \usefont{LYG}{bigygoth}{m}{n} \papername
   }
  }\hfill%	
  \raisebox{12pt}{
   \textbf{
    \footnotesize 
    \paperlocation
   }
  }\\
  \vspace*{0.1in}
  \rule[0pt]{\textwidth}{0.5pt}\\
  {\small 
    VOL.\MakeUppercase{\roman{volumeno}}
    \ldots No. \arabic{issueno}
   } \hfill 
   \MakeUppercase{\small  19. december 2023} 
   \hfill {\small }\\
  \rule[6pt]{\textwidth}{1.2pt}
  \end{center}
  \pagestyle{plain}
}

\def\ps@plain{%
  \renewcommand\@oddfoot{}%					% empty recto footer
  \let\@evenfoot\@oddfoot						% empty verso footer
  \renewcommand\@evenhead
  {\parbox{\textwidth}{\vspace*{4pt}
  {\small VOL.\MakeUppercase{\roman{volumeno}}\ldots No.\arabic{issueno}}\hfill\normalfont\textbf{\headername}\quad\MakeUppercase{\textit\today}\hfill\textrm{\thepage}\\
  \rule{\textwidth}{0.5pt}
  \vspace*{12pt}}}%
  \let\@oddhead\@evenhead
}
		

\newcommand\headline[1]{
  {\fontfamily{lmdh}\selectfont
    \begin{center} #1\\ %
    \rule[3pt]{0.4\hsize}{0.5pt}\\ \end{center} \par
  }
}

\newcommand\byline[2]{
  {\fontfamily{lmdh}\selectfont
  \begin{center} #1 \\%
  {\footnotesize\bf af \MakeUppercase{#2}} \\ %
  \rule[3pt]{0.4\hsize}{0.5pt}\\ \end{center} \par
  }
}


\newcommand\closearticle{{\begin{center}\rule[6pt]{\hsize}{1pt}\vspace*{-16pt}
			\rule{\hsize}{0.5pt}\end{center}}}



\begin{document}


\maketitle
\fontfamily{phv}\selectfont

\begin{multicols}{2}
\byline{Back in the game}{Fernándo Sanchez}
\subsection*{All Good Things come to an End}
\saymaria{Kan vi tage bare en tur mere?}\\
\sayfernando{Vi har været i gang i 8 timer! Tror du ikke at den falder af?}\\
\saymaria{Visse vasse... Her, tag 2 mere af de blå, og en powernap... Så går jeg ud og laver en lille snack til os, og så tager vi 2. halvvej bagefter}\\
\sayfernando{2.?}\\
\saymaria{Well...} sagde hun, iførte sig sin korte natkjole og gik imod lejlighedens lille køkken imens hun talte på fingrene. Imens nærmest besvimede jeg på sofaen som jeg indtil for 2 minutter havde brugt som rampe.\\
Jeg nåede at sove i omtrent så lang tid som det tager at tilberede en omgang beef nachos, for da jeg vågnede af lyden fra en meget larmende motorcykel der stoppede nede på gaden, var Maria på vej ind i stuen med en bakkefuld beef nachos ingredienser.\\
\saymaria{Ej for helvede, nu nåede jeg ikke at lave guaca'en} stressede hun, og skyndte sig tilbage ud i køkkenet.\\
Jeg fik en fornemmelse af at noget vigtigt skulle til at ske, som blev forstærket da Maria kom løbende tilbage ind i stuen med en skålfuld igredienser til guacamole. Imens Maria havde haft travlt i køkkenet og jeg havde sovet, havde de blå piller også været flittige og nu var i hvert fald en lille del af mig lysvågen. Idet jeg rejste resten af kroppen op i sofaen, kunne jeg høre et kraftigt brøl fra trappen.

\saymartin{Hvem fanden har du nu knaldet med Maria?}\\
\sayfernando{Hvem er det?}\\
\saymaria{Det er min kæreste} sagde Maria, som nu havde taget stilling i lænestolen i hjørnet af stuen, og sad som skulle hendes yndlings TV-serie til at begynde.\\
\saymaria{Han er psykopat!} tilføjede hun med et kæmpe smil.\\\\
\sayfernando{Erhvervspsykopat?} spurgte jeg håbefuldt.\\
\saymaria{(NEJ!)} rystede hun på hovedet, med øjne der skinnede af forventning, og skovlede en håndful ostede nachos indenbords.\\
\sayfernando{Oh}\\

Med det samme blev vores lille tosomhed i lejligheden brudt, idet en kæmpe økse brød igennem hoveddøren og nogle sekunder efter kom en blodindsølet balrog i julemandskostume bragende igennem døren, svingende vildt omkring sig med en lige så blodindsmurt økse.\\

\saymartin{Du er kraftedeme en død mand, gadedreng!} sagde det enorme menneske og tonsede imod mig med øksen truende hævet over hovedet. I mellemtiden nåede Maria ét-eller-andet slags klimaks i lænestolen, og lod sig falde tilbage med et veltrilfreds smil.\\

Jeg forestiller mig at du, kære læser, tilhører det bedre borgerskab og er udstyret med en over-gennemsnits begavelse, så jeg vil ikke tale ned til dig og forsøge at give en suspenseful beretning af begivenhederne der fulgte; eftersom jeg skriver disse linjer har du jo nok regnet ud at jeg overlevede mødet med den enorme fyr. Som en public service, vil jeg i stedet (i figur \ref{fig:diffs}) påpege nogle subtile forskelle imellem reglerne (eller mangel på samme) der gælder i det velkendte rituelle gadeslagsmål og kampen for overlevelse imod vilde økse-svingende voldspsykopater. 





\subsection*{Alperne}

Jeg stod foran føtex kl. 17:56, og som punktligheden selv kom Anna på slaget 18:00.\\
\sayanna{Den her slutrunde har overgået alle vores forventninger! Konkurrencen med især OOCMO@Danske Bank har tvunget os til at gå spritnye veje denne jul, og resultatet er ikke til at tage fejl af: vi vidste at vi havde en vinder med Musical Contest; det er et efterhånden velkendt format for os, og vores brugere har denne gang mestendels stemt rationelt, men opbakningen til Team Contest har været FORMIDABEL! Vi har ramt et format som har lokket de mest drevne - men samtidigt også mest korrumperede - sider af den menneskelige natur frem. Vi har set folk der har snydt i Find Nissen, og tilmed været nødt til at diskvalificere et hold på et tidspunkt!}


\sayfernando{Det må jeg give jer, det lyder som noget af en succes!}


\sayanna{Kæmpe! Men vi er kun lige kommet til Alperne, og set vores første kategori 1 bjerg. I morgen (tirsdag) har vi et kategori 2 bjerg, og onsdag forventer vi at finde vinderen af Team Contest når vi går \hspace{5px} \textbf{Hors catégorie}. Vi har set en kraftpræstation fra nogle meget dedikerede indvidualister som alle ligger meget lunt til at vinde den gule trøje, men det er vigtigt for mig at påpege, at såvel Team Contest som den indviduelle konkurrence er \textbf{PIV-åbne}, for dem der stadigvæk har saft i benene!}\\

\sayfernando{Hold da op! Er alle holdene stadigvæk med i løbet om førstepladsen?}\\
\sayanna{Nej! Vi har indset at der er meget stor forskel på modenheden af de deltagende teams, og vi må desværre sande at nogle teams bare ikke har niveauet til at deltage på det her plan. Til næste slutrunde-arrangement under OOCMO, kommer vi til at stille større krav til de deltagende hold, og vi kommer til at frasortere de teams og deltagere som ikke har performet…}\\

Jeg mærkede pludselig noget spidst stikke mig i siden
\saysune{\textbf{Hænderne op, det er politiet! Jesper ... mmmm... Jespersen, du er anholdt for æresløst at stikke af fra en kamp til døden, i værste fald strafbart ved Blodørnen}} erklærede en dyb stemne højt. Selv uden de åbenlyse ledetråde, havde bunden af vanvid i mandens stemme ikke efterladt mig med nogen tvivl om dens ejermand. De forbipasserende vendte sig for at se hvad der skete.\\

\saysune{\textbf{Hvil I blot trygt kære Pøbel! Ordensmagten har styr på sagerne, og vil nu kropsvisitere truslen og sikre at den ikke har sprængstoffer eller andre skadelige sager på sig!}}\\

Jeg vidste jo nok hvad han var ude efter, så jeg sparede ham besværet og rakte ham glasset med stryknin.

\saysune{\textbf{Der ser i Befolkning! Det Gode vinder endnu engang… for det har Bamse selv bestemt!}}\\

Da jeg vendte mig for at kigge efter ham, kunne jeg se CMO’en gadedrengeløbe sig muntert igennem de travle julemasser, med pilleglasset triumferende rejst i vejret.


\byline{Reportage fra en krigszone}{Thorkild Tjärnsten}
Havde jeg for 2 måneder siden bedt dig, kære læser, om at male mig et billede af Vesterbro en uge inden Juleaften, er jeg sikker på at det billede havde indeholdt letpåklædte skønheder fra alle verdenshjørner, glade gadesælgere og fornøjede forretningsfolk på vej hjem til kernefamilien efter en hård dag på arbejdet og en velfortjent optankning på overskuds-kontoen fra førnævnte. Billedet havde næppe indeholdt finansfolk på motorcykler, julemænd med flækøkser, gader indsmurt i blod og udtrådte avocadoer, og skræmte beboere der kigger magtesløse til bag køkkenvinduerne imens politiet ingen steder er at finde, men det er nu engang den virkelighed vi har levet i de sidste par uger.\\
Vi lavede en hurtig voxpop med én af de få lovlydige mennesker vi mødte på Istedgade.

\sayother{(\textbf{Artur Jävnström - studerende} ) Den her situation er så mega uholdbar! Jeg kom til København for at studere, møde nye mennesker og bede kvinder på Snapchat om at sende mig billeder af deres bryster; hvis jeg ville have blodsindsmurte voldsjulemænd og skræmte prostituerede var jeg jo blevet i Randers! Mine venner og jeg fra studiet holdte julefrokost i weekenden, og det var umuligt at støve blæs af både den ene og anden art op. Uhørt! Og hvor er politiet? Er det for pokker ikke os de bliver betalt for at beskytte? Vi er cremen af Danmarks ungdom! Det er os der kommer til at finansiere det her velfærdsgilde lige om lidt! Min fætter der er færdselsbetjent i Holbæk siger at alle senior-officerer i den københavnske politi-styrke er taget til Lanzarote på ubestemt tid!?!}


\byline{Blod-mysteriet}{Susan Saxing}
Natten til mandag blev endnu en ung mand der på ulyksagelig vis havde mistet alt sit blod, fundet død foran Emdrupparkens Idrætsanlæg. Derved er dødstallet oppe på 7 siden de makabre fund begyndte for ca. 3 uger siden. Solsikke Tidendes udsendte tog en snak med junior kriminalbetjent Noah Golt Spalt-Schläger.\\

\sayother{Jeg vil godt have at du opgører mig som \textbf{Kriminalkommissær Golt Spalt-Schläger}. Vil du gøre det? Jeg er nemlig sikker på at ``dem med stjerner på skuldrene'' nok skal belønne mig med en lille forfremmelse når jeg har knækket den her sag lige om lidt. Ser du... jeg har ``Scherlocket''... Jeg spurgte migselv: ``hvad ville Scherlock gøre?'', og så gik jeg ud og snakkede med beboerne i området, og det har så sandelig båret pote. Jeg har fra meget pålidelig kilde, at der kort tid før de gamle mænd begyndte at tabe blodet, er blevet rejst \textbf{5G antenner} i området. Coincidence? ``I think not my dear Watson!''}


\end{multicols}{2}

\begin{figure}
	\centering
\begin{tabular}{ | l | r| r | } 
  \hline
  & \textbf{Gadeslagsmål} & \textbf{Julemand} \\ 
  \hline
  \hline
  Slag, spark, tramp & \tabdo{Go for it} & \tabdo{Go for it} \\
  \hline
  cell7 & \tabdo{cell8} & \tabdont{cell9} \\ 
  \hline
\end{tabular}
	
\caption{Do's and don'ts i gadeslagsmål vs. kamp for overlevelse imod psykopat-julemænd}
\label{fig:diffs}
\end{figure}



\end{document}